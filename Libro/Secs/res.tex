% ------------------------------------------------------------------------
% ------------------------------------------------------------------------
% ------------------------------------------------------------------------
%                                Resumen
% ------------------------------------------------------------------------
% ------------------------------------------------------------------------
% ------------------------------------------------------------------------
\chapter*{Resumen}

\footnotesize{
\begin{description}
  \item[T�tulo:] Definici�n de una infraestructura cloud de alta disponibilidad en un entorno distribuido para el despliegue de una plataforma IoT\astfootnote{Trabajo de grado}
  \item[Autor:] Jose David Rojas Aguilar \asttfootnote{Facultad de Ingenier�as F�sico-Mec�nicas. Escuela de Ingenier�a de sistemas e inform�tica. Director: Gabriel Rodrigo Pedraza Ferreira, Doctor en Ciencias de la Computaci�n.}
  \item[Palabras Clave:] Internet de las cosas, Escalabilidad, Smart Campus, Infraestructura, Cloud, Alta Disponibilidad.
  \item[Descripci�n:] El internet de las cosas (IoT) ha tenido un gran impacto en los �ltimos a�os dentro de la sociedad en diferentes contextos. Una de sus aplicaciones m�s importantes es la mejora de la calidad de vida y el apoyo en el proceso de aprendizaje de los estudiantes dentro de las diferentes universidades. Un reto dentro del IoT es tener una infraestructura capaz de soportar una cantidad masiva de dispositivos enviando informaci�n de forma constante. \\
  
  Este trabajo de investigaci�n se busca dise�ar una arquitectura software para el despliegue de una plataforma IoT en una infraestructura cloud de alta disponibilidad en un entorno distribuido. El proyecto inici� con una fase de exploraci�n con el fin de identificar las herramientas disponibles en el mercado. Luego se definieron unos criterios de selecci�n a nivel t�cnico, unos requisitos de los artefactos a desplegar, un conjunto de m�tricas y un conjunto de pruebas para evaluar el desempe�o de la arquitectura plateada. Despu�s se defini� una arquitectura la cual fue evaluada, en una primera oportunidad, por las pruebas dise�adas durante el proyecto y, finalmente, en una prueba de integraci�n en un caso de uso de la plataforma IoT.
\end{description}}\normalsize
% ------------------------------------------------------------------------ 