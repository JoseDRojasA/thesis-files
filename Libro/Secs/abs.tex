% ------------------------------------------------------------------------
% ------------------------------------------------------------------------
% ------------------------------------------------------------------------
%                                Abstract
% ------------------------------------------------------------------------
% ------------------------------------------------------------------------
% ------------------------------------------------------------------------
\chapter*{Abstract}

\footnotesize{
\begin{description}
  \item[Title:] Definition of a cloud infrastructure of high availability in a distributed environment for the deployment of an iot platform\astfootnote{Bachelor Thesis}
  \item[Author:] Jose David Rojas Aguilar\asttfootnote{Faculty of Physical-Mechanical Engineering. School of Systems Engineering and Informatics. Advisor: Gabriel Rodrigo Pedraza Ferreira, PhD in Computer science}
  \item[Keywords:] Infrastructure, Cloud, High availability, Definition, Cluster.
  \item[Description:] The Internet of Things (IoT) has had a strong impact in recent years within society in different contexts. One of its most important applications is the improvement of the quality of life and the support in the learning process of the students in different universities. A challenge in the IoT is to have an infrastructure capable of supporting a massive amount of devices sending information constantly. \\
  
  This research work looks for design software architecture for the deployment of an IoT platform in a high availability cloud infrastructure in a distributed environment. The project began with an exploration phase in order to identify the tools available in the market. Then technical selection criteria were defined, requirements of the artifacts to be deployed, a set of metrics and a set of tests to evaluate the performance of the proposed architecture. An architecture was defined, which was evaluated, at a first opportunity, by the tests designed during the project and, finally, in an integration test in a case of use of the IoT platform.
\end{description}}\normalsize
% ------------------------------------------------------------------------ 