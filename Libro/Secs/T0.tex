% ------------------------------------------------------------------------
% ------------------------------------------------------------------------
% ------------------------------------------------------------------------
%                              Introducci�n
% ------------------------------------------------------------------------
% ------------------------------------------------------------------------
% ------------------------------------------------------------------------

\nnchapter{Introducci�n}
% ------------------------------------------------------------------------
% ------------------------------------------------------------------------


\noindent La expansi�n de internet y los grandes avances la comunicaci�n entre diferentes dispositivos ha permitido el surgimiento de un nuevo paradigma tecnol�gico: Internet de las cosas (IoT). IoT se define como la infraestructura mundial para la sociedad de la informaci�n, que permite poner diferentes servicios a dispositivos interconectados entre s�, gracias a las tecnolog�as de la informaci�n y comunicaci�n, permitiendo la extracci�n y procesamiento masivo de datos. \\

El internet de las cosas ha tenido un impacto significativo en contextos de diferentes magnitudes. Grandes empresas como Amazon, Microsoft, Google, Cisco, Philips e Intel han invertido millones en investigaci�n, generando diferentes servicios enfocados al hogar, empresas e incluso ciudades. Barcelona es una ciudad pionera en el uso de IoT en sus calles, por ejemplo, la l�nea 9 del metro de Barcelona se ha actualizado con ascensores inteligentes que utilizan los datos en tiempo real para adaptarse a las necesidades de los viajeros lo cual ha logrado disminuir las aglomeraciones y el consumo de energ�a para 30 millones de pasajeros al a�o\citep{BARCELONA2019}. \\

Uno de los campos de aplicaci�n del IoT se encuentra en las universidades con el fin de mejorar la calidad de vida y apoyar el proceso de formaci�n en los estudiantes. La implementaci�n de una soluci�n IoT suele ser una tarea muy compleja que requiere de una infraestructura hardware y software capaz de soportar una cantidad masiva de dispositivos enviando informaci�n de forma continua por lo cual una caracter�stica fundamental dentro de una infraestructura IoT es la alta disponibilidad. \\

En este trabajo de investigaci�n se presenta el dise�o de una infraestructura para el despliegue de una plataforma IoT en una infraestructura cloud de alta disponibilidad en un entorno distribuido con el fin de proveer un entorno que pueda soportar una gran cantidad de dispositivos conectados enviado datos de forma masiva. 
