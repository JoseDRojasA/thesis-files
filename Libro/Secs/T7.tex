% ------------------------------------------------------------------------
% ------------------------------------------------------------------------
% ------------------------------------------------------------------------
%                             Conclusiones
% ------------------------------------------------------------------------
% ------------------------------------------------------------------------
% ------------------------------------------------------------------------

\chapter{Trabajo futuro}

\begin{itemize}
	\item Explorar los diferentes plugins que ofrece la comunidad de Openshift Origin.
	\item Utilizar servidores con las caracter�sticas recomendadas por la documentaci�n de Openshift Origin.
%	\item Hacer uso de discos duros de estado s�lido con el fin de aumentar el rendimiento en los puntos en que es necesario acceder al disco, por ejemplo, las consultas hechas a la base de datos.
	\item Hacer uso de bases de datos distribuidas para poder realizar un escalamiento proporcional a la cantidad de pods. Tener en cuenta que el conector usado en el backend puede cambiar.
	\item Hacer la instalaci�n de los diferentes componentes directamente sobre las m�quinas y no recurrir a la virtualizaci�n como fue necesario durante la implementaci�n del prototipo.
%	\item Conectar los servidores por 10 Gigabit Ethernet en lugar de Gigabit Ethernet.
	\item Con el fin de minimizar los riesgos durante un fallo el�ctrico, ser�a de ayuda a�adir una UPS a la infraestructura.
	\item Hacer uso de un sistema de almacenamiento en la nube (AWS, GCP, Azure o DigitalOcean).
	\item Instalar un registro de Docker privado para guardar las im�genes.
	\item Implementar End to End testing.
\end{itemize}