
\chapter{Conclusiones}
% ------------------------------------------------------------------------
\noindent A partir de los desarrollos presentados y los resultados obtenidos en el presente trabajo de grado, se lleg� a las siguientes conclusiones:
% ------------------------------------------------------------------------ 
\begin{enumerate}
	\item La elecci�n de herramientas software adecuadas es fundamental durante la primera fase de un proyecto software ya que son estas las bases t�cnicas sobre las cuales se construir� el proyecto. Durante la primera fase del proyecto Se logr� identificar las diferentes herramientas en el mercado y escoger de estas las m�s adecuadas a partir de un conjunto de requerimientos extra�dos en conjunto con el equipo de desarrollo de la plataforma IoT.
	\item Una propiedad solicitada en un entorno de IoT es la alta disponibilidad ya que habr� una masiva cantidad de dispositivos conectados a la plataforma constantemente. Para soportar esta propiedad se dise�� una arquitectura en un ambiente distribuido que posteriormente se valid� por medio de las pruebas realizadas sobre un prototipo.
	\item Una caracter�stica importante dentro de una arquitectura software es su extensibilidad. Para simplificar el proceso de ingreso de nuevos componentes a la arquitectura original, se defini� una metodolog�a con un conjunto de convenciones para el despliegue de artefactos dentro de la infraestructura planteada.
	\item El despliegue continuo es un componente fundamental en el desarrollo de proyecto software modernos ya que automatiza tareas repetitivas del proceso de despliegue lo que permite a los diferentes miembros del equipo enfocarse en tareas de mayor valor para el proyecto. Durante el proyecto se integr� con �xito un sistema de despliegue continuo dentro de la arquitectura lo cual permiti� evidenciar la importancia de integrar una soluci�n de despliegue continuo en un proyecto software. 
\end{enumerate}