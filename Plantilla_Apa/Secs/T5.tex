% ------------------------------------------------------------------------
% ------------------------------------------------------------------------
% ------------------------------------------------------------------------
%                            Recomendaciones
% ------------------------------------------------------------------------
% ------------------------------------------------------------------------
% ------------------------------------------------------------------------

\chapter{Desarrollo del proyecto}
% ------------------------------------------------------------------------
\noindent En este Cap�tulo se presentar� como fue desarrollado el proyecto.
% ------------------------------------------------------------------------ 
\section{Ambientaci�n tecnol�gica}
\subsection{Investigaci\'on fundamentos te\'oricos relacionados con el proyecto (Tecnolog\'ias y est\'andares)}
Durante esta primera fase se investig� el estado del arte en cloud computing para poder identificar los conceptos comunes en las diferentes arquitecturas cloud enfocadas a IoT. De esta primera fase se encontr� la gesti�n de contenedores como factor com�n por lo cual la investigaci�n posterior sigui� este enfoque. Posteriormente se identificaron las tecnolog�as mas populares en IoT.

\begin{table}[H]
	\begin{minipage}{1\textwidth}
		\caption[Tecnolog�as disponibles en el mercado]{ \raggedright Tecnolog�as disponibles en el mercado}
		\label{tabla:Tecnolog�as disponibles en el mercado}
		\begin{center}
			\begin{tabular}{ p{5cm}   p{5cm}  p{5cm}  }
				\hline
				Sistema Operativo & Container Runtime & Orquestador de contenedores \\
				\hline
				CentOS & Cri-O & Apache Mesos \\
				CoreOS & Docker & Docker Swarm \\
				Red Hat Enterprise Linux & Linux VServer & Kontena \\
				Oracle Linux & LXD & Kubernetes \\
				Ubuntu Server & Rkt  & Nomad \\
				Windows Server & Windows Containers & Openstack Magnum x
				\\ \hline
			\end{tabular}
		\end{center}
	\end{minipage}
\end{table}

A medida que se desarroll� en proyecto, nuevas tecnolog�as fueron a�adidas al stack pero ya de eso se hablar� mas adelante en el documento.

A medida que se desarroll� en proyecto, nuevas tecnolog�as fueron a�adidas al stack pero ya de eso se hablar� mas adelante en el documento.

\subsection{Estudio}
Durante esta etapa se realiz� una revisi�n general sobre cada una de las tecnolog�as, lo cual sirvi� de base para la definir los criterios de selecci�n para realizar un primer filtro sobre las tecnolog�as a usar en la infraestructura en la siguiente etapa.

\begin{itemize}
	\item Licencia
	\item Stacks
	\item Soporte
	\item Uso libre en producci�n
\end{itemize}